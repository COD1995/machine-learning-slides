\documentclass[10pt,dvipsnames]{beamer}
\usepackage[T1]{fontenc}
\usepackage{libertinus}
\usepackage{amsmath}
\usepackage[most]{tcolorbox}

\usepackage{graphicx}

\usepackage{hyperref}
\hypersetup{
    colorlinks=true,
    linkcolor=blue,    % color of internal links
    urlcolor=blue,     % color of external URLs
    citecolor=blue     % color of citations
}

\usepackage{xcolor}  
\newcommand{\cb}[1]{{\color{CadetBlue}#1}}


\usetheme{Goettingen}
\setbeamertemplate{footline}[frame number]
\setbeamertemplate{navigation symbols}{}


\title{CSE574 Introduction to Machine Learning}
\subtitle{Machine Learning: Notation and Definitions}
\author{Jue Guo}
\institute{University at Buffalo}
\date{\today}

\begin{document}
\begin{frame}
    \titlepage
\end{frame}

\begin{frame}
    \frametitle{Outline}
    \tableofcontents
\end{frame}

\section{Notation}
\begin{frame}{Notation}
    Let's breifly revisit the mathematical notation we all learned at school.
\end{frame}

\subsection{Data Structure}
\begin{frame}{Data Structure}
    A \textbf{scalar} is a simple numerical value, like 15 or -3.25 . Variables or constants that take scalar values are denoted by an italic letter, like $x$ or $a$.
    
    A \textbf{vector} is an ordered list of scalar values, called attributes. We denote a vector as a bold character, for example, $\mathbf{x}$ or $\mathbf{w}$. 
    \begin{itemize}
    	\item Vectors can be visualized as arrows that point to some directions as well as points in a multi-dimensional space. 
    \end{itemize}
    Illustrations of three two-dimensional vectors, $\mathbf{a}=[2,3], \mathbf{b}=[-2,5]$, and $\mathbf{c}=[1,0]$ are given in the figure. 
	\begin{figure}
		\centering
		\includegraphics[width=0.7\textwidth]{imgs/notation_1}
		\caption{Three vectors visualized as directions and as points.}
		\label{fig:notation1}
	\end{figure}
\end{frame}

\begin{frame}
	We denote an attribute of a vector as an italic value with an index, like this: $w^{(j)}$ or $x^{(j)}$. The index $j$ denotes a specific \textbf{dimension} of the vector, the position of an attribute in the list. For instance, in the vector a shown in red in the figure, $a^{(1)}=2$ and $a^{(2)}=3$.
		\begin{figure}
		\centering
		\includegraphics[width=0.7\textwidth]{imgs/notation_1}
		\caption{Three vectors visualized as directions and as points.}
	\end{figure}
	The notation $x^{(j)}$ should not be confused with the power operator, such as the 2 in $x^{2}$ (squared) or 3 in $x^{3}$ (cubed). If we want to apply a power operator, say square, to an indexed attribute of a vector, we write like this: $\left(x^{(j)}\right)^{2}$.
	
	A variable can have two or more indices, like this: $x_{i}^{(j)}$ or like this $x_{i, j}^{(k)}$. For example, in neural networks, we denote as $x_{l, u}^{(j)}$ the input feature $j$ of unit $u$ in layer $l$.
\end{frame}

\begin{frame}
	A \textbf{matrix} is a rectangular array of numbers arranged in rows and columns. Below is an example of a matrix with two rows and three columns,
	$$
	\left[\begin{array}{ccc}
		2 & 4 & -3 \\
		21 & -6 & -1
	\end{array}\right]
	$$
	
	Matrices are denoted with bold capital letters, such as \(\mathbf{A}\) or \(\mathbf{W}\).
	
	A \textbf{set} is an unordered collection of unique elements. 
	\begin{itemize}
		\item We denote a set as a calligraphic capital character, for example, $\mathcal{S}$.
	\end{itemize}
	 \textcolor{red}{A set of numbers can be finite (include a fixed amount of values).} 
	 \begin{itemize}
	 	\item  In this case, it is denoted using accolades, for example, $\{1,3,18,23,235\}$ or $\left\{x_{1}, x_{2}, x_{3}, x_{4}, \ldots, x_{n}\right\}$. 
	 \end{itemize}
	
	 \textcolor{red}{A set can be infinite and include all values in some interval.}
	 \begin{itemize}
	 	\item If a set includes all values between $a$ and $b$, including $a$ and $b$, it is denoted using brackets as $[a, b]$. If the set doesn't include the values $a$ and $b$, such a set is denoted using parentheses like this: $(a, b)$. 
	 \end{itemize}

	 \begin{itemize}
	 	\item For example, the set $[0,1]$ includes such values as $0,0.0001,0.25,0.784,0.9995$, and 1.0. A special set denoted $\mathbb{R}$ includes all numbers from minus infinity to plus infinity.
	 \end{itemize}

\end{frame}

\end{document}